%%%%%%%%%%%%%%%%%
% This is an example CV created using altacv.cls (v1.1.4, 27 July 2018) written by
% LianTze Lim (liantze@gmail.com), based on the
% Cv created by BusinessInsider at http://www.businessinsider.my/a-sample-resume-for-marissa-mayer-2016-7/?r=US&IR=T
%
%% It may be distributed and/or modified under the
%% conditions of the LaTeX Project Public License, either version 1.3
%% of this license or (at your option) any later version.
%% The latest version of this license is in
%%    http://www.latex-project.org/lppl.txt
%% and version 1.3 or later is part of all distributions of LaTeX
%% version 2003/12/01 or later.
%%%%%%%%%%%%%%%%

%% If you want to use \orcid or the
%% academicons icons, add "academicons"
%% to the \documentclass options.
%% Then compile with XeLaTeX or LuaLaTeX.
% \documentclass[10pt,a4paper,academicons]{altacv}

%% Use the "normalphoto" option if you want a normal photo instead of cropped to a circle
% \documentclass[10pt,a4paper,normalphoto]{altacv}

\documentclass[10pt,a4paper]{altacv}

\usepackage{graphicx}

%% AltaCV uses the fontawesome and academicon fonts
%% and packages.
%% See texdoc.net/pkg/fontawecome and http://texdoc.net/pkg/academicons for full list of symbols.
%% When using the "academicons" option,
%% Compile with LuaLaTeX for best results. If you
%% want to use XeLaTeX, you may need to install
%% Academicons.ttf in your operating system's font %% folder.


% Change the page layout if you need to
\geometry{left=1cm,right=9cm,marginparwidth=6.8cm,marginparsep=1.2cm,top=1cm,bottom=1cm}

% Change the font if you want to.

% If using pdflatex:
\usepackage[utf8]{inputenc}
\usepackage[T1]{fontenc}
\usepackage[default]{lato}

% If using xelatex or lualatex:
% \setmainfont{Lato}

% Change the colours if you want to
\definecolor{VividPurple}{HTML}{008080}
\definecolor{SlateGrey}{HTML}{2E2E2E}
\definecolor{LightGrey}{HTML}{666666}
\colorlet{heading}{VividPurple}
\colorlet{accent}{VividPurple}
\colorlet{emphasis}{SlateGrey}
\colorlet{body}{LightGrey}

% Change the bullets for itemize and rating marker
% for \cvskill if you want to
\renewcommand{\itemmarker}{{\small\textbullet}}
\renewcommand{\ratingmarker}{\faCircle}

%% sample.bib contains your publications
\addbibresource{sample.bib}

\begin{document}
\name{ALESSANDRO TAMAI}
\tagline{Mathematician}
% Cropped to square from https://en.wikipedia.org/wiki/Marissa_Mayer#/media/File:Marissa_Mayer_May_2014_(cropped).jpg, CC-BY 2.0
\photo{4.5cm}{mod3}
\personalinfo{%
  % Not all of these are required!
  % You can add your own with \printinfo{symbol}{detail}
  Born in Treviso on February 24, 1995.   \linebreak \linebreak
  \email{tamaiatwork@gmail.com }
  \phone{+39 3469506415}
  \location{ Trieste , Italy  }
 % \homepage{https://yogs447.blogspot.com/}
  \linebreak
  \linkedin{https://www.linkedin.com/in/alessandro-tamai-6b8719203/}
%   \github{github.com/mmayer} % I'm just making this up though.
%   \orcid{orcid.org/0000-0000-0000-0000} % Obviously making this up too. If you want to use this field (and also other academicons symbols), add "academicons" option to \documentclass{altacv}
}

%% Make the header extend all the way to the right, if you want.
\begin{fullwidth}
\makecvheader
\end{fullwidth}

%% Depending on your tastes, you may want to make fonts of itemize environments slightly smaller
\AtBeginEnvironment{itemize}{\small}

%% Provide the file name containing the sidebar contents as an optional parameter to \cvsection.
%% You can always just use \marginpar{...} if you do
%% not need to align the top of the contents to any
%% \cvsection title in the "main" bar.




\

\cvsection[page1sidebar]{Education}

\cvevent{Master's degree in Mathematics} {   \includegraphics[scale=0.028]{logo} University of Trieste  }{2017-2020}{Trieste , Italy } 
\textsc{Final Mark}: 110/110 (summa cum laude)
\divider
\\
\cvevent{ Bachelor's degree in Mathematics}{   \includegraphics[scale=0.028]{logo} University of Trieste}{2014-2017}{Trieste , Italy }
\textsc{Final Mark}: 104/110
\begin{itemize}
\item   \includegraphics[scale=0.063]{UCL}  Erasmus program at "UCL: Universit\'e   Catholique de Louvain-La-Neuve": \cvevent{}{}{Sept. 2017- Feb. 2017}{ Louvain-la-Neuve , Belgium}
\end{itemize}
\divider
\\
\cvevent{Diploma of Perito Chimico}{Istituto Superiore Statale
"Giorgi-Fermi" }{2009-2014}{Treviso , Italy}
\textsc{Final Mark}: 78/ 100

% \divider

% \cvevent{Product Engineer}{Google}{23 June 1999 -- 2001}{Palo Alto, CA}

% \begin{itemize}
% \item Joined the company as employe \#20 and female employee \#1
% \item Developed targeted advertisement in order to use user's search queries and show them related ads
% \end{itemize}



%%%%%%%%%%%%%%%%%%%%%%%%%%%%%%%%%%%%%%%%%%
%%%%%%%%%%%%%%%%%%%%%%%%%%%%%%%%%%%%%%%%%%
%%%%%%%%%%%%%%%%%%%%%%%%%%%%%%%%%%%%%%%%%%
%%%%%%%%%%%%%%%%%%%%%%%%%%%%%%%%%%%%%%%%%%



\cvsection{Dissertation Work}
\cvevent{Bachelor Thesis}{  Introduzione alla Teoria delle Perturbazioni di Sistemi Hamiltoniani Autonomi }{}{}
\begin{itemize}
\item  Reminder on symplectic geometry and its relation with Hamiltonian systems.
\item Liouville-Arnol'd Theorem and its consequences.
\item Introduction to perturbation theory with a focus on  Poincar\'e's negative results.
\item  Summary of the historical contrast between ergotic theory and perturbation theory in  relation to Fermi's computational experiments.
\item KAM Theorem and its similarities with Liouville-Arnol'd Theorem.
\end{itemize}

\cvevent{Master Thesis}{Singular Solutions of Rolling Balls Model: a Topological View}{}{}
\begin{itemize}
\item Lie groups and semisimple Lie algebras classification with a focus on the $G_2$ group.
\item Division algebras: quaternions and their relations with the rotation of the three dimensional space; the octonions and their split form.
\item The group $G_2$ as automorphism group of octonions and split octonions as conseguence of the Lie group classification.
\item Sub-Riemannian geometry and  singular solutions in relation to the rolling balls problem.
\item Overview on the most recent results: relations between the model and quaternions, octonions and the $G_2$ group.  

\item Presentation of the research work's results: description of the geometric spaces of singular solutions, their topological properties and their relation with lens spaces.



\end{itemize}
	
\
\\


\cvsection[page2sidebar]{University Projects }
\begin{itemize}
\item "Flux Analysis for Competing Firms": 

\textit{(Final exam of "Stochastic Modelling and Simulation" course)}

Given a set of consumers and firms distributed on a geographic region, can we estimate how the consumers' choice depends on some given parameters? In the project I try to answer this question by design a mathematical model based on discrete Markov's chains and analyzing the simulations made by it python implementation.
\end{itemize}



\cvsection{Scientific Work}

\cvevent{"Singular Solutions of Rolling Balls Problem" }{  \includegraphics[scale=0.018]{SISSA}  Geometric Structures Research Seminar}{January 12, 2021}{SISSA, Trieste, Italy}
\begin{itemize}
\item  Speaker.
\end{itemize}
https://sites.google.com/view/geometric-structures/
\
\\
\divider
\cvevent{"Singular Solutions of Rolling Balls Problem" }{  \includegraphics[scale=0.065]{UPC}  Geometry and Dynamical System Group Seminar}{April 19 2021}{UPC, Barcelona, Spain}
\begin{itemize}
\item  Speaker.
\end{itemize}



\cvsection{Scholarships}

\cvevent{ }{   \includegraphics[scale=0.018]{SISSA} Postgraduate Fellowship }{April 2021 - July 2021}{SISSA, Trieste, Italy}




%%%%%%%%%%%%%%%%%%%%%%%%%%%%%%%%%%%%%%%%%%
%%%%%%%%%%%%%%%%%%%%%%%%%%%%%%%%%%%%%%%%%%
%%%%%%%%%%%%%%%%%%%%%%%%%%%%%%%%%%%%%%%%%%
%%%%%%%%%%%%%%%%%%%%%%%%%%%%%%%%%%%%%%%%%%
%%%%%%%%%%%%%%%%%%%%%%%%%%%%%%%%%%%%%%%%%%


\cvsection{Work Experience}


%%%%%%%%%%%%%%%%%%%%%%%%%%%%%%%%%%%%%%%%%%
%%%%%%%%%%%%%%%%%%%%%%%%%%%%%%%%%%%%%%%%%%
%%%%%%%%%%%%%%%%%%%%%%%%%%%%%%%%%%%%%%%%%%
\cvevent{Exerciser of  "Geometria 1"  (  double assignment, ch. A,B  ) }{Universita' degli Studi di Trieste}{ October 2019- January 2020  }{Trieste (Italia)}

\begin{itemize}
\item  Exercises sessions on the course's topics.
\item  Correction of practice tests.
\end{itemize}


\cvevent{Exerciser of "Geometria 3B" }{Universita' degli Studi di Trieste}{March 2019- June 2019 }{Trieste (Italia)}

\begin{itemize}
\item  Exercises sessions on the course's topics.

\end{itemize}


%%%%%%%%%%%%%%%%%%%%%%%%%%%%%%%%%%%%%%%%%%



\cvevent{Exerciser of  "Analisi 1"  (double assignment, ch. A,B  ) }{Universita' degli Studi di Trieste}{        October 2018- January 2019}{Trieste (Italia)}

\begin{itemize}
\item  Exercises sessions on the course's topics.
\item  Correction of practice tests.
\end{itemize}

%%%%%%%%%%%%%%%%%%%%%%%%%%%%%%%%%%%%%%%%%%




%%%%%%%%%%%%%%%%%%%%%%%%%%%%%%%%%%%%%%%%%%



%%%%%%%%%%%%%%%%%%%%%%%%%%%%%%%%%%%%%%%%%%




\cvevent{MaxMeyer Promoter}{Leroy Merlin}{March 2014 - June 2014 }{San Biagio di Callalta, Italy}
\begin{itemize}
\item Preparation of paints through dedicated instrumentation.
\item Shop assistant work in paints, hardware and building material aisle.  

\end{itemize}


%%%%%%%%%%%%%%%%%%%%%%%%%%%%%%%%%%%%%%%%%%
%%%%%%%%%%%%%%%%%%%%%%%%%%%%%%%%%%%%%%%%%%
%%%%%%%%%%%%%%%%%%%%%%%%%%%%%%%%%%%%%%%%%%
%%%%%%%%%%%%%%%%%%%%%%%%%%%%%%%%%%%%%%%%%%




\clearpage


\end{document}
